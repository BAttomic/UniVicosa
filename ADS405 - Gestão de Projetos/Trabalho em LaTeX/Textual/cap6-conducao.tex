\chapter{CONDUÇÃO DO PROJETO}

Este capítulo descreve os procedimentos de execução e controle do projeto, detalhando como as atividades planejadas serão conduzidas, monitoradas e ajustadas durante todo o ciclo de desenvolvimento da plataforma de ingressos online. A condução eficaz do projeto requer processos estruturados de gestão que garantam alinhamento entre execução e planejamento \cite{PMI:2021}.

\section{Governança e Estrutura de Gestão}

A governança do projeto estabelece a estrutura de autoridade, responsabilidade e processos decisórios que guiarão a execução \cite{Kerzner:2022}.

\subsection{Estrutura Organizacional}

O projeto adota uma estrutura organizacional matricial balanceada, onde o Gerente de Projeto possui autoridade sobre o planejamento e controle, enquanto os especialistas técnicos mantêm autonomia para decisões técnicas dentro de suas áreas de expertise. Esta estrutura promove equilíbrio entre controle de gestão e flexibilidade técnica, essencial para projetos de desenvolvimento ágil.

\textbf{Comitê de Direcionamento:} Composto pelo orientador acadêmico, Gerente de Projeto e representantes dos stakeholders principais. Reúne-se quinzenalmente para revisão de progresso, aprovação de mudanças de escopo significativas e resolução de impedimentos de alto nível.

\textbf{Equipe de Desenvolvimento:} Auto-organizada para decisões técnicas cotidianas, com autonomia para escolher as melhores abordagens para implementação dentro dos requisitos e restrições estabelecidos.

\subsection{Processos Decisórios}

As decisões são categorizadas em três níveis:

\textbf{Decisões Estratégicas:} Mudanças de escopo, arquitetura fundamental, orçamento e cronograma macro. Requerem aprovação do Comitê de Direcionamento.

\textbf{Decisões Táticas:} Escolhas de tecnologias específicas, abordagens de implementação e priorização de backlog. Decididas pelo Gerente de Projeto em conjunto com especialistas técnicos.

\textbf{Decisões Operacionais:} Detalhes de implementação, estruturas de código e testes. Decididas autonomamente pela equipe de desenvolvimento durante sprints.

\section{Gestão de Comunicação}

A comunicação eficaz é fundamental para o sucesso do projeto, garantindo alinhamento entre stakeholders e transparência sobre progresso e desafios \cite{PMI:2021}.

\subsection{Canais de Comunicação}

\textbf{Reuniões Presenciais/Virtuais:}
\begin{itemize}
    \item Daily Standup: Diariamente, 15 minutos, toda a equipe técnica
    \item Sprint Planning: A cada 2 semanas, 4 horas, toda a equipe
    \item Sprint Review: A cada 2 semanas, 2 horas, equipe + stakeholders
    \item Sprint Retrospective: A cada 2 semanas, 1,5 horas, toda a equipe
    \item Reunião do Comitê: Quinzenalmente, 1 hora, comitê de direcionamento
\end{itemize}

\textbf{Ferramentas Assíncronas:}
\begin{itemize}
    \item Slack ou Microsoft Teams: Comunicação rápida e informal
    \item Jira ou Azure DevOps: Gestão de backlog, tarefas e bugs
    \item Confluence ou Notion: Documentação colaborativa
    \item GitHub/GitLab: Revisões de código e discussões técnicas
\end{itemize}

\subsection{Relatórios e Documentação}

\textbf{Relatório Semanal de Progresso:} Enviado toda segunda-feira pelo Gerente de Projeto, contendo:
\begin{itemize}
    \item Resumo do progresso da semana anterior
    \item Marcos atingidos e entregas completadas
    \item Impedimentos identificados e ações de resolução
    \item Próximos passos e atividades críticas
    \item Indicadores-chave de projeto (cronograma, orçamento, qualidade)
\end{itemize}

\textbf{Dashboard de Projeto:} Painel visual atualizado em tempo real com métricas como:
\begin{itemize}
    \item Burndown chart do sprint atual
    \item Velocity da equipe (story points completados por sprint)
    \item Quantidade de testes passando/falhando
    \item Cobertura de código
    \item Dívida técnica acumulada
\end{itemize}

\textbf{Atas de Reuniões:} Documentação de todas as reuniões formais, incluindo decisões tomadas, ações definidas e responsáveis.

\section{Gestão de Riscos}

A gestão de riscos é um processo contínuo ao longo do projeto, envolvendo identificação, análise, planejamento de respostas e monitoramento \cite{PMI:2021}.

\subsection{Processo de Identificação}

Novos riscos são identificados continuamente através de:
\begin{itemize}
    \item Retrospectivas de sprint (lições aprendidas)
    \item Análise de métricas técnicas (performance degradando, dívida técnica crescendo)
    \item Feedback de stakeholders
    \item Monitoramento de tecnologias e dependências externas
\end{itemize}

\subsection{Registro e Priorização}

Todos os riscos são documentados no Registro de Riscos contendo:
\begin{itemize}
    \item Descrição do risco
    \item Probabilidade de ocorrência (Baixa/Média/Alta)
    \item Impacto potencial (Baixo/Médio/Alto/Crítico)
    \item Exposição ao risco (Probabilidade × Impacto)
    \item Estratégia de resposta (Mitigar/Transferir/Aceitar/Evitar)
    \item Ações preventivas planejadas
    \item Responsável pelo monitoramento
    \item Indicadores de alerta precoce
\end{itemize}

\subsection{Estratégias de Resposta}

Para os riscos de maior prioridade identificados no Capítulo 3, as seguintes estratégias serão implementadas:

\textbf{R001 - Picos de Demanda Extrema:}
\begin{itemize}
    \item \textit{Mitigação:} Implementação de sistema de filas virtuais com Redis desde o início
    \item \textit{Mitigação:} Configuração agressiva de auto-scaling (escala a 60\% de CPU)
    \item \textit{Mitigação:} Utilização de CDN para distribuir carga de assets estáticos
    \item \textit{Monitoramento:} Testes de carga progressivos a cada 3 sprints
\end{itemize}

\textbf{R002 - Falhas na Integração de Pagamentos:}
\begin{itemize}
    \item \textit{Mitigação:} Integração com 2 gateways independentes desde o início
    \item \textit{Mitigação:} Implementação de circuit breaker e fallback automático
    \item \textit{Mitigação:} Testes extensivos em ambiente sandbox antes de produção
    \item \textit{Monitoramento:} Alertas automáticos para taxa de erro > 1\% em transações
\end{itemize}

\textbf{R003 - Fraudes e Falsificação:}
\begin{itemize}
    \item \textit{Mitigação:} QR Codes dinâmicos que se atualizam a cada 30 segundos
    \item \textit{Mitigação:} Vinculação obrigatória ingresso-usuário com validação de identidade
    \item \textit{Mitigação:} Logs detalhados de tentativas de validação para análise forense
    \item \textit{Monitoramento:} Dashboard de detecção de padrões anômalos
\end{itemize}

\textbf{R004 - Conectividade no Local do Evento:}
\begin{itemize}
    \item \textit{Mitigação:} Modo offline robusto com cache local de ingressos válidos
    \item \textit{Mitigação:} Sincronização inteligente quando conectividade é restaurada
    \item \textit{Mitigação:} Backup de conectividade via 4G/5G em dispositivos de check-in
    \item \textit{Monitoramento:} Testes em condições reais em evento piloto
\end{itemize}

\section{Gestão de Mudanças}

Mudanças de escopo, requisitos ou arquitetura seguem um processo formal de controle para avaliar impactos e garantir decisões informadas \cite{PMI:2021}.

\subsection{Processo de Solicitação de Mudança}

\begin{enumerate}
    \item \textbf{Solicitação:} Qualquer stakeholder pode solicitar mudança através de formulário padronizado descrevendo a mudança proposta e justificativa
    
    \item \textbf{Análise de Impacto:} Gerente de Projeto e equipe técnica analisam:
    \begin{itemize}
        \item Impacto no escopo, cronograma e orçamento
        \item Impacto na arquitetura e integração com componentes existentes
        \item Riscos introduzidos pela mudança
        \item Valor agregado versus custo de implementação
    \end{itemize}
    
    \item \textbf{Classificação:}
    \begin{itemize}
        \item \textbf{Mudança Menor:} Sem impacto em cronograma/orçamento, aprovada pelo GP
        \item \textbf{Mudança Moderada:} Impacto < 5\% em cronograma/orçamento, aprovada pelo Comitê
        \item \textbf{Mudança Maior:} Impacto ≥ 5\%, requer aprovação do orientador acadêmico
    \end{itemize}
    
    \item \textbf{Decisão:} Aprovação ou rejeição com justificativa documentada
    
    \item \textbf{Implementação:} Se aprovada, mudança é incorporada ao backlog e priorizada
    
    \item \textbf{Comunicação:} Todos os stakeholders são notificados da decisão e impactos
\end{enumerate}

\subsection{Baseline e Controle de Versão}

O projeto estabelece baselines (linhas de base) formais em marcos importantes:
\begin{itemize}
    \item \textbf{Baseline de Requisitos:} Após aprovação do Capítulo 3 (Semana 4)
    \item \textbf{Baseline de Arquitetura:} Após aprovação do design arquitetural (Semana 6)
    \item \textbf{Baseline de Escopo:} Antes do início do desenvolvimento core (Semana 8)
\end{itemize}

Mudanças pós-baseline requerem processo formal de controle. O código-fonte segue versionamento semântico (MAJOR.MINOR.PATCH) para rastreamento claro de mudanças.

\section{Gestão da Qualidade}

A qualidade é assegurada através de processos contínuos de verificação, validação e melhoria \cite{Pressman:2019}.

\subsection{Garantia da Qualidade (QA)}

\textbf{Revisões de Código:} Todo código produzido passa por peer review antes de merge:
\begin{itemize}
    \item Revisão de no mínimo 1 desenvolvedor sênior
    \item Checklist de qualidade: legibilidade, performance, segurança, testes
    \item Validação automatizada de estilo de código (linters)
\end{itemize}

\textbf{Definition of Done (DoD):} Critérios obrigatórios para considerar uma funcionalidade completa:
\begin{itemize}
    \item Código implementado e revisado
    \item Testes unitários escritos e passando (cobertura ≥ 80\%)
    \item Testes de integração escritos e passando
    \item Documentação técnica atualizada
    \item Sem vulnerabilidades críticas ou altas detectadas
    \item Funcionalidade demonstrada e aprovada pelo Product Owner
\end{itemize}

\textbf{Auditorias de Qualidade:} Realizadas a cada 4 sprints pelo Analista de QA:
\begin{itemize}
    \item Revisão de conformidade com padrões de codificação
    \item Análise de dívida técnica acumulada
    \item Validação de cobertura de testes
    \item Verificação de documentação atualizada
\end{itemize}

\subsection{Controle da Qualidade}

\textbf{Testes Automatizados Contínuos:}
\begin{itemize}
    \item Execução automática de testes unitários a cada commit
    \item Execução de testes de integração a cada push para branch develop
    \item Testes E2E executados diariamente em ambiente de homologação
    \item Relatórios de cobertura gerados automaticamente
\end{itemize}

\textbf{Análise Estática de Código:}
\begin{itemize}
    \item SonarQube integrado ao pipeline de CI/CD
    \item Quality Gates definindo limites aceitáveis de dívida técnica
    \item Bloqueio automático de merges que introduzam vulnerabilidades críticas
\end{itemize}

\textbf{Testes de Performance:}
\begin{itemize}
    \item Testes de carga progressivos (1k, 5k, 10k usuários) a cada 3 sprints
    \item Benchmark de tempos de resposta de APIs críticas
    \item Análise de uso de recursos (CPU, memória, rede)
\end{itemize}

\section{Gestão de Recursos Humanos}

A gestão eficaz da equipe é essencial para manter motivação, produtividade e qualidade das entregas.

\subsection{Alocação e Nivelamento}

\textbf{Matriz de Habilidades:} Documentação das competências de cada membro, identificando:
\begin{itemize}
    \item Especialidades técnicas (linguagens, frameworks, ferramentas)
    \item Experiência em domínios (pagamentos, segurança, UX)
    \item Nível de proficiência (iniciante, intermediário, avançado, especialista)
\end{itemize}

\textbf{Alocação Dinâmica:} Membros da equipe são alocados a tarefas considerando:
\begin{itemize}
    \item Match entre habilidades requeridas e disponíveis
    \item Equilíbrio de carga de trabalho (evitar sobrecarga ou subutilização)
    \item Oportunidades de desenvolvimento profissional (pair programming)
    \item Dependências entre tarefas e especialização necessária
\end{itemize}

\subsection{Desenvolvimento da Equipe}

\textbf{Pair Programming:} Prática encorajada especialmente para:
\begin{itemize}
    \item Tarefas complexas ou críticas
    \item Transferência de conhecimento entre membros
    \item Onboarding de novos integrantes
\end{itemize}

\textbf{Tech Talks Internos:} Apresentações quinzenais de 30 minutos onde membros compartilham:
\begin{itemize}
    \item Novas tecnologias e ferramentas descobertas
    \item Soluções criativas para problemas enfrentados
    \item Lições aprendidas de bugs ou incidentes
\end{itemize}

\textbf{Retrospectivas Focadas em Time:} Além das retrospectivas técnicas, sessões trimestrais focadas em:
\begin{itemize}
    \item Dinâmica e colaboração da equipe
    \item Identificação de frustrações e impedimentos não técnicos
    \item Celebração de sucessos e reconhecimento
\end{itemize}

\section{Gestão de Aquisições e Fornecedores}

O projeto depende de diversos fornecedores e serviços externos que devem ser adequadamente gerenciados.

\subsection{Seleção de Fornecedores}

\textbf{Provedores de Nuvem:} Seleção baseada em critérios como:
\begin{itemize}
    \item Custo total de propriedade (TCO) para cargas de trabalho previstas
    \item Disponibilidade de serviços gerenciados necessários
    \item Conformidade com regulamentações (LGPD)
    \item Qualidade de suporte técnico e SLAs oferecidos
\end{itemize}

\textbf{Gateway de Pagamento:} Avaliação baseada em:
\begin{itemize}
    \item Taxa de aprovação de transações (conversion rate)
    \item Taxas por transação e custos recorrentes
    \item Certificações de segurança (PCI DSS Level 1)
    \item Suporte a métodos de pagamento (PIX, cartão)
    \item Qualidade de documentação e facilidade de integração
\end{itemize}

\subsection{Contratos e SLAs}

Todos os contratos com fornecedores críticos incluem:
\begin{itemize}
    \item Service Level Agreements (SLAs) claros com métricas mensuráveis
    \item Cláusulas de penalidade por descumprimento de SLA
    \item Direitos de portabilidade de dados
    \item Responsabilidades em caso de incidentes de segurança
    \item Processos de suporte e escalação
\end{itemize}

\subsection{Monitoramento de Performance}

Performance de fornecedores é monitorada continuamente:
\begin{itemize}
    \item Disponibilidade de APIs e serviços externos
    \item Tempos de resposta de integrações
    \item Taxa de erro em chamadas a serviços externos
    \item Custos reais versus estimados
\end{itemize}

Reuniões trimestrais de revisão com fornecedores principais para:
\begin{itemize}
    \item Análise de cumprimento de SLAs
    \item Discussão de incidentes e melhorias
    \item Negociação de otimizações de custo
\end{itemize}

\section{Gestão de Integração}

A gestão de integração garante que todos os componentes do projeto trabalhem de forma coesa e coordenada \cite{PMI:2021}.

\subsection{Coordenação de Atividades}

\textbf{Sincronização de Sprints:} Todas as equipes (quando aplicável) trabalham em sprints sincronizados de 2 semanas, facilitando:
\begin{itemize}
    \item Integração contínua de componentes desenvolvidos em paralelo
    \item Cerimônias compartilhadas (planning, review, retrospective)
    \item Resolução coordenada de dependências
\end{itemize}

\textbf{Integração Técnica Contínua:} Pipeline de CI/CD garante que:
\begin{itemize}
    \item Código de múltiplos desenvolvedores é integrado diariamente
    \item Conflitos de integração são detectados rapidamente
    \item Testes de integração validam comunicação entre componentes
    \item Builds são gerados automaticamente para ambientes de teste
\end{itemize}

\subsection{Gestão de Dependências}

\textbf{Matriz de Dependências:} Documentação de todas as dependências entre:
\begin{itemize}
    \item Tarefas e atividades do projeto
    \item Componentes de software (microsserviços)
    \item Fornecedores e serviços externos
\end{itemize}

\textbf{Resolução de Bloqueios:} Processo expedito para desbloquear tarefas:
\begin{itemize}
    \item Identificação durante daily standup
    \item Escalação imediata ao Gerente de Projeto
    \item Mobilização de recursos para resolução prioritária
    \item Comunicação proativa a stakeholders afetados
\end{itemize}

\section{Encerramento de Sprints e Fases}

Cada sprint e fase principal do projeto possui procedimentos formais de encerramento.

\subsection{Encerramento de Sprint}

\textbf{Sprint Review:} Demonstração de incremento funcional para stakeholders:
\begin{itemize}
    \item Apresentação de funcionalidades completadas
    \item Coleta de feedback imediato
    \item Ajuste de prioridades para próximos sprints
\end{itemize}

\textbf{Sprint Retrospective:} Reflexão da equipe sobre processo:
\begin{itemize}
    \item O que funcionou bem (manter)
    \item O que não funcionou (eliminar)
    \item O que pode ser melhorado (experimentar)
    \item Definição de ações concretas de melhoria
\end{itemize}

\textbf{Atualização de Métricas:} Registro de indicadores do sprint:
\begin{itemize}
    \item Velocity alcançada (story points completados)
    \item Bugs encontrados e corrigidos
    \item Cobertura de testes final
    \item Dívida técnica adicionada ou reduzida
\end{itemize}

\subsection{Encerramento de Fases}

Ao final de cada fase principal (Planejamento, Desenvolvimento Core, Testes, Implantação):

\textbf{Revisão de Entregáveis:} Validação formal de que todos os entregáveis planejados foram completados e atendem aos critérios de aceitação.

\textbf{Lições Aprendidas:} Documentação estruturada de:
\begin{itemize}
    \item Práticas que funcionaram bem e devem ser mantidas
    \item Problemas enfrentados e como foram resolvidos
    \item Recomendações para fases futuras ou projetos similares
\end{itemize}

\textbf{Aprovação Formal:} Assinatura de termo de aceite pelo Comitê de Direcionamento autorizando progressão para a próxima fase.

Este conjunto integrado de processos de condução assegura que o projeto seja executado de forma controlada, com visibilidade contínua de progresso, riscos e qualidade, permitindo ajustes proativos para garantir sucesso na entrega.
