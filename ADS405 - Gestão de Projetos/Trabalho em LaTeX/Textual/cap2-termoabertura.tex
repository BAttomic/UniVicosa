\chapter{ANÁLISE DE STAKEHOLDERS E TERMO DE ABERTURA DO PROJETO}

\section{Análise dos \textit{Stakeholders}}

A plataforma de venda de ingressos online representa uma solução tecnológica que impacta diversos stakeholders no ecossistema de eventos. Para compreender adequadamente as necessidades e expectativas dos usuários, foi realizada uma análise detalhada dos stakeholders, identificando suas características, desafios e objetivos específicos.

Os stakeholders da plataforma podem ser categorizados em diferentes grupos, incluindo organizadores de eventos de diversos portes, consumidores finais que adquirem ingressos, operadores de controle de acesso nos eventos, e stakeholders secundários como provedores de pagamento e equipes técnicas. Cada grupo possui características específicas que demandam abordagens diferenciadas no desenvolvimento da solução.

\subsection{Personas}

As personas representam arquétipos dos usuários principais da plataforma de ingressos, baseadas em pesquisa e análise do mercado de eventos. Foram identificadas três personas principais que representam os diferentes perfis de stakeholders da solução.

\subsubsection{Persona 1: João Silva - Organizador de Eventos}

João Silva, 35 anos, é produtor de eventos culturais e musicais de médio porte. Organiza de 5 a 10 eventos por ano, com capacidade entre 500 e 5.000 pessoas, incluindo shows, festivais e eventos corporativos.

\textbf{Características principais:}
\begin{itemize}
    \item Experiência de 8 anos no setor de eventos e entretenimento
    \item Foco em eficiência operacional e controle rigoroso de custos
    \item Preocupação constante com segurança e prevenção de fraudes
    \item Valoriza tecnologias que ofereçam relatórios detalhados e controle total
\end{itemize}

\textbf{Principais desafios:}
\begin{itemize}
    \item Sistemas atuais falham durante picos de demanda em lançamentos de eventos populares
    \item Dificuldade no controle efetivo de cambismo e circulação de ingressos falsificados
    \item Falta de relatórios detalhados e em tempo real sobre vendas e perfil do público
    \item Complexidade na gestão de diferentes tipos de ingressos e lotes promocionais
\end{itemize}

\textbf{Objetivos específicos:}
\begin{itemize}
    \item Plataforma confiável que suporte alta demanda sem degradação de performance
    \item Ferramentas eficazes de controle de acesso e prevenção de fraudes
    \item Relatórios analíticos em tempo real sobre vendas, público e receita
    \item Interface administrativa intuitiva para gestão autônoma de eventos
\end{itemize}

\subsubsection{Persona 2: Maria Santos - Compradora de Ingressos}

Maria Santos, 28 anos, é analista de marketing em uma empresa de tecnologia e frequentadora assídua de eventos culturais, shows e festivais. Compra ingressos online mensalmente e valoriza experiências digitais fluidas e seguras.

\textbf{Características principais:}
\begin{itemize}
    \item Usuária experiente de plataformas digitais e e-commerce
    \item Valoriza rapidez, segurança e transparência no processo de compra
    \item Utiliza principalmente smartphone para pesquisa e aquisição de ingressos
    \item Compartilha experiências em redes sociais e influencia decisões de amigos
\end{itemize}

\textbf{Principais desafios:}
\begin{itemize}
    \item Filas virtuais longas e instáveis que frequentemente resultam em erro
    \item Processos de compra confusos, demorados e com múltiplas etapas desnecessárias
    \item Insegurança quanto à autenticidade dos ingressos e proteção de dados pessoais
    \item Dificuldade para acessar e gerenciar ingressos adquiridos após a compra
\end{itemize}

\textbf{Objetivos específicos:}
\begin{itemize}
    \item Experiência de compra rápida, intuitiva e sem frustrações
    \item Segurança total nas transações financeiras e proteção de dados
    \item Acesso fácil e organizado aos ingressos adquiridos
    \item Transparência sobre taxas, políticas de cancelamento e informações do evento
\end{itemize}

\subsubsection{Persona 3: Carlos Mendes - Operador de Portaria}

Carlos Mendes, 42 anos, trabalha na equipe de controle de acesso de diversos tipos de eventos há mais de 10 anos. É responsável pela validação de ingressos na entrada e pelo controle de fluxo de pessoas, trabalhando frequentemente sob pressão em horários de pico.

\textbf{Características principais:}
\begin{itemize}
    \item Experiência consolidada em controle de acesso e segurança de eventos
    \item Necessita de ferramentas simples, eficientes e de fácil operação
    \item Trabalha em condições adversas: ruído, multidões e pressão temporal
    \item Valoriza soluções que funcionem offline ou com conectividade limitada
\end{itemize}

\textbf{Principais desafios:}
\begin{itemize}
    \item Dificuldade na identificação rápida de ingressos falsificados ou adulterados
    \item Sistemas lentos de validação que geram filas extensas na entrada
    \item Falta de informações em tempo real sobre status e validade dos ingressos
    \item Dependência de conectividade que pode falhar em momentos críticos
\end{itemize}

\textbf{Objetivos específicos:}
\begin{itemize}
    \item Validação instantânea e segura de ingressos através de QR Code
    \item Interface simples e otimizada para uso em dispositivos móveis
    \item Funcionamento confiável mesmo com conectividade limitada
    \item Informações claras sobre status do ingresso e dados do portador
\end{itemize}

\section{Mapa de Empatia}

O mapa de empatia é uma ferramenta utilizada para compreender de forma profunda as perspectivas, sentimentos e comportamentos dos principais stakeholders da plataforma de ingressos online. A visualização permite identificar não apenas o que os usuários fazem, mas também o que pensam, sentem e experienciam em seu contexto de uso da plataforma.

\IfFileExists{img/mapaempatia.png}{%
\begin{figure}[H]
    \centering
    \includegraphics[width=\textwidth]{img/mapaempatia.png}
    \caption{Estrutura do mapa de empatia aplicada aos stakeholders da plataforma de ingressos online.}
    \label{fig:mapaempatia}
\end{figure}
}{%
\begin{center}
\fbox{\parbox{0.9\textwidth}{
\textbf{Mapa de Empatia Visual}\\[0.5em]
O mapa de empatia visual dos stakeholders da plataforma de ingressos online será desenvolvido em ferramenta gráfica específica e inserido posteriormente nesta seção.\\[0.5em]
\textit{Arquivo planejado: img/mapaempatia\_ingressos.png}
}}
\end{center}
}

O mapa de empatia desenvolvido para este projeto organiza as informações dos três perfis principais identificados: João Silva (Organizador de Eventos), Maria Santos (Compradora de Ingressos) e Carlos Mendes (Operador de Portaria). Para cada persona, são detalhados os seguintes aspectos:

\subsection{Análise Empática dos Stakeholders}

\textbf{João Silva - Organizador de Eventos:}
\begin{itemize}
    \item \textbf{O que vê:} Dashboards de vendas, relatórios de capacidade, interfaces administrativas, feedback de participantes
    \item \textbf{O que escuta:} Reclamações sobre sistemas instáveis, alertas de segurança, demandas por transparência
    \item \textbf{O que pensa e sente:} Preocupação com fraudes, ansiedade durante lançamentos, pressão por resultados
    \item \textbf{Dores:} Sistemas que falham em picos de demanda, falta de controle sobre cambismo, relatórios insuficientes
    \item \textbf{Ganhos:} Plataforma confiável, ferramentas de controle eficazes, insights em tempo real
\end{itemize}

\textbf{Maria Santos - Compradora de Ingressos:}
\begin{itemize}
    \item \textbf{O que vê:} Interfaces de compra, filas virtuais, confirmações de pagamento, ingressos digitais
    \item \textbf{O que escuta:} Experiências de outros compradores, alertas de eventos, notificações do sistema
    \item \textbf{O que pensa e sente:} Frustração com processos lentos, insegurança sobre autenticidade, pressa para garantir ingressos
    \item \textbf{Dores:} Filas instáveis, processos confusos, insegurança sobre fraudes, dificuldade de acesso posterior
    \item \textbf{Ganhos:} Compra rápida e segura, acesso fácil aos ingressos, transparência no processo
\end{itemize}

\textbf{Carlos Mendes - Operador de Portaria:}
\begin{itemize}
    \item \textbf{O que vê:} QR Codes, telas de validação, multidões na entrada, dispositivos móveis
    \item \textbf{O que escuta:} Instruções de supervisores, reclamações do público, alertas do sistema
    \item \textbf{O que pensa e sente:} Pressão por agilidade, preocupação com segurança, estresse em horários de pico
    \item \textbf{Dores:} Sistemas lentos, dificuldade para identificar fraudes, dependência de conectividade
    \item \textbf{Ganhos:} Validação instantânea, interface simples, funcionamento offline confiável
\end{itemize}

Esta análise empática fundamenta o desenvolvimento de uma solução verdadeiramente centrada no usuário, garantindo que cada funcionalidade e decisão de design esteja alinhada com as necessidades reais dos stakeholders identificados.

\section{Requisitos de Alto Nível}

Com base na análise detalhada dos stakeholders e considerando os objetivos da plataforma de ingressos online, foram identificados os seguintes requisitos de alto nível:

\subsection{Requisitos Funcionais de Alto Nível}

\textbf{RF001 - Sistema de Gestão de Eventos:}
A plataforma deve permitir o cadastro completo de eventos, incluindo informações detalhadas, configuração de ingressos, lotes promocionais, capacidade e políticas específicas.

\textbf{RF002 - Processo de Compra Otimizado:}
O sistema deve oferecer um fluxo de compra intuitivo e eficiente, com carrinho de compras, múltiplos métodos de pagamento e confirmação automática por e-mail.

\textbf{RF003 - Ingressos Digitais Seguros:}
A solução deve gerar ingressos digitais com QR Code dinâmico e único, vinculados à conta do usuário para prevenir fraudes e cambismo.

\textbf{RF004 - Controle de Acesso:}
A plataforma deve incluir sistema de validação rápida e segura de ingressos na portaria, com funcionamento offline e sincronização posterior.

\subsection{Requisitos Não-Funcionais de Alto Nível}

\textbf{RNF001 - Escalabilidade:}
O sistema deve suportar picos de demanda de até 10.000 usuários simultâneos sem degradação significativa de performance.

\textbf{RNF002 - Segurança:}
A plataforma deve implementar padrões de segurança PCI DSS para transações financeiras e criptografia de dados sensíveis.

\textbf{RNF003 - Usabilidade:}
A interface deve ser responsiva e otimizada para dispositivos móveis, com tempo de carregamento inferior a 2 segundos.

\section{Termo de Abertura do Projeto}

\subsection{Justificativa do Projeto}

A indústria de eventos enfrenta desafios críticos na comercialização de ingressos online. Os sistemas atuais apresentam deficiências significativas em três pilares fundamentais: escalabilidade durante picos de demanda, segurança contra fraudes e experiência do usuário. Esta situação resulta em perdas financeiras substanciais para organizadores, frustração recorrente para consumidores e vulnerabilidades de segurança que comprometem a integridade dos eventos. O desenvolvimento de uma solução moderna, robusta e centrada no usuário é essencial para elevar os padrões de qualidade e confiabilidade do setor.

\subsection{Objetivos do Projeto}

\subsubsection{Objetivo Geral}
Projetar e desenvolver um sistema web completo e robusto para venda de ingressos online, priorizando segurança, escalabilidade e usabilidade, com o propósito de otimizar a gestão de eventos para organizadores e proporcionar uma experiência de compra transparente e confiável para consumidores.

\subsubsection{Objetivos Específicos}
\begin{itemize}
    \item Desenvolver módulo de identidade e acesso seguro para diferentes perfis de usuário
    \item Implementar interface administrativa completa para gestão autônoma de eventos
    \item Construir fluxo de compra otimizado com integração a gateway de pagamento moderno
    \item Projetar sistema de ingressos digitais seguros utilizando QR Codes dinâmicos
    \item Criar portal do cliente para visualização e gerenciamento de ingressos
    \item Desenvolver solução de controle de acesso eficiente para validação na portaria
\end{itemize}

\subsection{Escopo do Projeto}

\subsubsection{Dentro do Escopo}
\begin{itemize}
    \item Plataforma web responsiva compatível com desktop, tablet e smartphone
    \item Sistema de cadastro detalhado de eventos com múltiplas categorias de ingressos
    \item Implementação de filas virtuais para gerenciamento de alta demanda
    \item Processo de compra completo com confirmação automática por e-mail
    \item Integração com API de pagamento seguindo normas PCI DSS de segurança
    \item Geração de ingressos digitais com QR Code que se atualiza periodicamente
    \item Aplicação simplificada para controle de acesso com validação em tempo real
\end{itemize}

\subsubsection{Fora do Escopo}
\begin{itemize}
    \item Desenvolvimento de aplicativos móveis nativos para Android e iOS
    \item Funcionalidades de mercado secundário para revenda de ingressos
    \item Sistema interativo para escolha de assentos marcados
    \item Módulos de marketing, afiliados e programas de fidelidade
    \item Painéis analíticos complexos e relatórios financeiros avançados
    \item Recursos de interação social e sistemas de avaliação
\end{itemize}

\subsection{Stakeholders Identificados}

\textbf{Stakeholders Primários:}
\begin{itemize}
    \item Organizadores de eventos (pequeno, médio e grande porte)
    \item Consumidores finais (compradores de ingressos)
    \item Operadores de portaria e controle de acesso
\end{itemize}

\textbf{Stakeholders Secundários:}
\begin{itemize}
    \item Equipe de desenvolvimento e manutenção
    \item Provedores de gateway de pagamento
    \item Fornecedores de infraestrutura de nuvem
    \item Órgãos reguladores de proteção de dados
\end{itemize}

\subsection{Cronograma Macro}

\begin{itemize}
    \item \textbf{Fase 1 - Análise e Planejamento:} 4 semanas
    \item \textbf{Fase 2 - Desenvolvimento do Core:} 12 semanas
    \item \textbf{Fase 3 - Desenvolvimento Complementar:} 8 semanas
    \item \textbf{Fase 4 - Testes e Validação:} 4 semanas
    \item \textbf{Fase 5 - Implantação e Go-live:} 2 semanas
    \item \textbf{Duração Total:} 30 semanas (aproximadamente 7,5 meses)
\end{itemize}

\subsection{Recursos Necessários}

\textbf{Recursos Humanos:}
\begin{itemize}
    \item 1 Gerente de Projeto
    \item 2 Desenvolvedores Full-Stack Sênior
    \item 1 Especialista em Segurança e Pagamentos
    \item 1 Designer UX/UI
    \item 1 Analista de Testes e QA
\end{itemize}

\textbf{Recursos Tecnológicos:}
\begin{itemize}
    \item Infraestrutura de nuvem escalável (AWS/Azure/GCP)
    \item CDN para distribuição global de conteúdo
    \item Gateway de pagamento certificado PCI DSS
    \item Ferramentas de desenvolvimento e CI/CD
    \item Sistemas de monitoramento e observabilidade
\end{itemize}

\textbf{Recursos Financeiros:}
\begin{itemize}
    \item Orçamento total estimado: R\$ 450.000,00
    \item Recursos humanos: R\$ 320.000,00 (71\%)
    \item Infraestrutura e tecnologia: R\$ 80.000,00 (18\%)
    \item Contingência e imprevistos: R\$ 50.000,00 (11\%)
\end{itemize}

\subsection{Riscos Iniciais}

\textbf{Riscos Técnicos:}
\begin{itemize}
    \item Picos de demanda excedendo capacidade de infraestrutura planejada
    \item Complexidade na integração com múltiplos gateways de pagamento
    \item Vulnerabilidades de segurança em transações financeiras
\end{itemize}

\textbf{Riscos de Negócio:}
\begin{itemize}
    \item Mudanças regulatórias em pagamentos digitais (PIX, cartões)
    \item Concorrência de plataformas estabelecidas no mercado
    \item Resistência de organizadores em migrar de sistemas atuais
\end{itemize}

\textbf{Riscos de Projeto:}
\begin{itemize}
    \item Atrasos no desenvolvimento devido à complexidade técnica
    \item Indisponibilidade de recursos especializados em momentos críticos
    \item Mudanças de escopo durante o desenvolvimento
\end{itemize}

\subsection{Critérios de Sucesso}

\textbf{Critérios Técnicos:}
\begin{itemize}
    \item Tempo de resposta inferior a 2 segundos para 95\% das requisições
    \item Disponibilidade superior a 99,9\% (máximo 8,76 horas de indisponibilidade por ano)
    \item Capacidade para processar 10.000 transações simultâneas
    \item Taxa de erro inferior a 0,1\% em transações de pagamento
\end{itemize}

\textbf{Critérios de Segurança:}
\begin{itemize}
    \item Taxa de fraude inferior a 0,1\% do volume total de transações
    \item Certificação PCI DSS Level 1 para processamento de pagamentos
    \item Zero vazamentos de dados pessoais ou financeiros
\end{itemize}

\textbf{Critérios de Experiência do Usuário:}
\begin{itemize}
    \item Taxa de abandono no processo de compra inferior a 15\%
    \item Satisfação do usuário superior a 4,5/5,0 em pesquisas pós-compra
    \item Tempo médio de finalização de compra inferior a 3 minutos
    \item Taxa de sucesso na validação de ingressos superior a 99,5\%
\end{itemize}

\textbf{Critérios de Negócio:}
\begin{itemize}
    \item Processamento de pelo menos 100.000 ingressos no primeiro ano
    \item Adesão de pelo menos 50 organizadores de eventos nos primeiros 6 meses
    \item ROI positivo a partir do 18º mês de operação
    \item Crescimento de 25\% no volume de transações a cada trimestre
\end{itemize}
