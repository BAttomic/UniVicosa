\chapter{ENGENHARIA DE REQUISITOS}

\section{Requisitos Funcionais}

Com base na análise dos stakeholders e no Termo de Abertura, foram identificados os seguintes requisitos funcionais para a plataforma de ingressos online:

\textbf{RF001 - Sistema de Autenticação e Autorização}
\begin{itemize}
    \item Cadastro de usuários com validação de e-mail
    \item Suporte a perfis: Organizador, Comprador e Operador de Portaria
    \item Autenticação segura com JWT e OAuth 2.0
\end{itemize}

\textbf{RF002 - Gestão de Eventos}
\begin{itemize}
    \item Cadastro completo de eventos com informações detalhadas
    \item Configuração de múltiplos tipos de ingressos e lotes promocionais
    \item Upload de imagens e descrições dos eventos
    \item Definição de capacidade e políticas específicas
\end{itemize}

\textbf{RF003 - Processo de Compra}
\begin{itemize}
    \item Carrinho de compras com seleção de quantidade e tipos
    \item Integração com gateway de pagamento (PIX e cartão de crédito)
    \item Sistema de filas virtuais para gerenciar alta demanda
    \item Confirmação automática por e-mail
\end{itemize}

\textbf{RF004 - Ingressos Digitais Seguros}
\begin{itemize}
    \item Geração de QR Code único e dinâmico para cada ingresso
    \item Vinculação direta à conta do usuário comprador
    \item Prevenção de screenshots e compartilhamento indevido
    \item Atualização periódica do código para evitar clonagem
\end{itemize}

\textbf{RF005 - Controle de Acesso}
\begin{itemize}
    \item Aplicação para validação rápida via leitura de QR Code
    \item Funcionamento offline com sincronização posterior
    \item Registro de entrada e controle de capacidade em tempo real
    \item Interface otimizada para uso em dispositivos móveis
\end{itemize}

\textbf{RF006 - Portal do Cliente}
\begin{itemize}
    \item Visualização de todos os ingressos adquiridos
    \item Histórico completo de compras
    \item Reenvio de comprovantes e ingressos por e-mail
    \item Gestão de dados pessoais e preferências
\end{itemize}

\section{Requisitos Não Funcionais}

\textbf{RNF001 - Performance e Escalabilidade}
\begin{itemize}
    \item Tempo de resposta inferior a 2 segundos para 95\% das requisições
    \item Suporte a 10.000 usuários simultâneos sem degradação
    \item Disponibilidade superior a 99,9\% (máximo 8,76h de indisponibilidade/ano)
    \item Auto-scaling automático baseado na demanda
\end{itemize}

\textbf{RNF002 - Segurança}
\begin{itemize}
    \item Conformidade com padrões PCI DSS para processamento de pagamentos \cite{PCIDSS:2022}
    \item Criptografia de dados sensíveis em trânsito e em repouso
    \item Proteção contra ataques do OWASP Top 10 \cite{OWASP:2021}
    \item Auditoria completa de todas as transações financeiras
\end{itemize}

\textbf{RNF003 - Usabilidade}
\begin{itemize}
    \item Interface responsiva para desktop, tablet e smartphone
    \item Navegação intuitiva com máximo 3 cliques para finalizar compra
    \item Suporte aos principais navegadores (Chrome, Firefox, Safari, Edge)
    \item Acessibilidade seguindo diretrizes WCAG 2.1 \cite{WCAG:2018}
\end{itemize}

\textbf{RNF004 - Confiabilidade}
\begin{itemize}
    \item Backup automático de dados críticos
    \item Recuperação de desastres em menos de 4 horas
    \item Monitoramento 24/7 com alertas automáticos
    \item Logs detalhados para auditoria e troubleshooting
\end{itemize}

\section{Histórias de Usuário}

\textbf{US001 - Comprar Ingresso}
\begin{itemize}
    \item \textbf{Como} comprador, \textbf{quero} selecionar um evento e finalizar a compra rapidamente, \textbf{para} garantir meu ingresso sem frustrações.
    \item \textbf{Critérios de Aceitação:} Processo completo em menos de 3 minutos, confirmação por e-mail, ingresso disponível imediatamente.
\end{itemize}

\textbf{US002 - Validar Entrada}
\begin{itemize}
    \item \textbf{Como} operador de portaria, \textbf{quero} escanear QR Codes rapidamente, \textbf{para} evitar filas na entrada do evento.
    \item \textbf{Critérios de Aceitação:} Validação em menos de 2 segundos, funcionamento offline, informações claras na tela.
\end{itemize}

\textbf{US003 - Gerenciar Evento}
\begin{itemize}
    \item \textbf{Como} organizador, \textbf{quero} acompanhar vendas em tempo real, \textbf{para} tomar decisões estratégicas durante a comercialização.
    \item \textbf{Critérios de Aceitação:} Dashboard atualizado a cada 5 minutos, relatórios exportáveis, alertas de capacidade.
\end{itemize}

\section{Riscos do Projeto}

\textbf{R001 - Picos de Demanda Extrema}
\begin{itemize}
    \item \textbf{Descrição:} Sistema pode não suportar demanda massiva em lançamentos de eventos populares
    \item \textbf{Probabilidade:} Alta | \textbf{Impacto:} Crítico
    \item \textbf{Mitigação:} Implementar filas virtuais, CDN global e auto-scaling agressivo
\end{itemize}

\textbf{R002 - Falhas na Integração de Pagamentos}
\begin{itemize}
    \item \textbf{Descrição:} Problemas com gateways podem impedir finalização de compras
    \item \textbf{Probabilidade:} Média | \textbf{Impacto:} Alto
    \item \textbf{Mitigação:} Múltiplos provedores, fallback automático e testes de carga extensivos
\end{itemize}

\textbf{R003 - Fraudes e Falsificação}
\begin{itemize}
    \item \textbf{Descrição:} Tentativas de burlar sistema com ingressos falsos ou cambismo
    \item \textbf{Probabilidade:} Média | \textbf{Impacto:} Alto
    \item \textbf{Mitigação:} QR Codes dinâmicos, machine learning para detecção de padrões suspeitos
\end{itemize}

\textbf{R004 - Conectividade no Local do Evento}
\begin{itemize}
    \item \textbf{Descrição:} Internet instável pode afetar validação de ingressos na portaria
    \item \textbf{Probabilidade:} Média | \textbf{Impacto:} Médio
    \item \textbf{Mitigação:} Modo offline robusto, sincronização inteligente e backup de conectividade
\end{itemize}

\section{Restrições}

As restrições do projeto definem as limitações e condições obrigatórias que devem ser respeitadas durante o desenvolvimento da plataforma \cite{Sommerville:2016}.

\textbf{Restrições Tecnológicas}
\begin{itemize}
    \item Conformidade obrigatória com padrões PCI DSS Level 1 para processamento de pagamentos \cite{PCIDSS:2022}
    \item Proteção contra vulnerabilidades do OWASP Top 10 \cite{OWASP:2021}
    \item Acessibilidade seguindo diretrizes WCAG 2.1 nível AA \cite{WCAG:2018}
    \item Ausência de aplicativos móveis nativos -- solução exclusivamente web responsiva
    \item Utilização obrigatória de tecnologias de nuvem com auto-scaling automático
\end{itemize}

\textbf{Restrições Legais e Regulatórias}
\begin{itemize}
    \item Conformidade total com a Lei Geral de Proteção de Dados (LGPD) \cite{LGPD:2018}
    \item Atendimento às normas do Código de Defesa do Consumidor para e-commerce
    \item Adequação às regulamentações de eventos públicos e privados
    \item Compliance com normas de segurança financeira do Banco Central do Brasil
\end{itemize}

\textbf{Restrições Orçamentárias}
\begin{itemize}
    \item Orçamento total limitado a R\$ 450.000{,}00
    \item Distribuição fixa: 71\% recursos humanos, 18\% infraestrutura, 11\% contingência
    \item Impossibilidade de contratação de recursos adicionais além do planejado
    \item Aprovação obrigatória para mudanças de escopo que impactem custos
\end{itemize}

\textbf{Restrições Temporais}
\begin{itemize}
    \item Prazo fixo de 30 semanas para desenvolvimento completo
    \item Datas de entrega das fases não negociáveis conforme cronograma acadêmico
    \item Marcos obrigatórios de entrega conforme calendário da disciplina
\end{itemize}

\textbf{Restrições Operacionais}
\begin{itemize}
    \item Sistema de controle de acesso deve funcionar offline com sincronização posterior
    \item Dependência de conectividade limitada em locais de eventos
    \item Interface deve ser operável em condições adversas (ruído, multidões, pressão temporal)
    \item Disponibilidade mínima de 99{,}9\% mesmo durante picos de demanda
\end{itemize}
