\chapter{RESULTADOS ESPERADOS}

Este capítulo apresenta os resultados esperados ao término do projeto de desenvolvimento da plataforma de ingressos online, estabelecendo as entregas concretas, as melhorias proporcionadas aos stakeholders e os impactos projetados nas dimensões técnica, operacional e de negócio. Os resultados estão alinhados aos objetivos específicos definidos no Capítulo 1 e fundamentados nas melhores práticas de engenharia de software \cite{Pressman:2019, Sommerville:2016}.

\section{Entregas Técnicas}

\subsection{Plataforma Web Completa e Operacional}

Ao término do projeto, espera-se a entrega de uma plataforma web completa, responsiva e funcional, composta pelos seguintes módulos integrados:

\textbf{1. Sistema de Autenticação e Gestão de Identidades}

O sistema permitirá que usuários se cadastrem e autentiquem de forma segura, com suporte a múltiplos perfis (comprador, organizador, operador de portaria). A implementação utilizará JSON Web Tokens (JWT) para gestão de sessões e OAuth 2.0 para integração opcional com provedores externos (Google, Facebook). Espera-se que o sistema atinja:

\begin{itemize}
    \item Tempo de resposta inferior a 500ms para autenticação
    \item Taxa de sucesso superior a 99,9\% em processos de login
    \item Zero vulnerabilidades críticas relacionadas a autenticação e autorização
    \item Conformidade total com requisitos de proteção de credenciais do OWASP \cite{OWASP:2021}
\end{itemize}

\textbf{2. Módulo de Gestão de Eventos}

Interface administrativa completa permitindo que organizadores cadastrem eventos com informações detalhadas, configurem múltiplos tipos de ingressos, definam lotes promocionais com preços diferenciados e façam upload de imagens promocionais. O módulo entregará:

\begin{itemize}
    \item CRUD completo de eventos com validação de dados
    \item Suporte a categorias de ingressos ilimitadas por evento
    \item Sistema de lotes com controle automático de disponibilidade
    \item Dashboard em tempo real com estatísticas de vendas
    \item Capacidade de processar 100 eventos ativos simultaneamente
\end{itemize}

\textbf{3. Sistema de Comercialização e Pagamentos}

Fluxo completo de compra desde a seleção de ingressos até a confirmação de pagamento, incluindo carrinho de compras resiliente, integração com gateway de pagamento certificado PCI DSS, e sistema de filas virtuais para alta demanda. Resultados esperados:

\begin{itemize}
    \item Tempo médio de finalização de compra inferior a 3 minutos
    \item Taxa de abandono de carrinho inferior a 15\%
    \item Suporte a PIX e cartão de crédito com processamento em tempo real
    \item Taxa de sucesso em transações superior a 99\%
    \item Capacidade de processar 10.000 transações simultâneas sem degradação
\end{itemize}

\textbf{4. Geração de Ingressos Digitais Seguros}

Sistema de geração de ingressos digitais utilizando QR Codes dinâmicos e únicos, vinculados diretamente à conta do comprador para prevenir falsificações e cambismo. Espera-se entregar:

\begin{itemize}
    \item Geração instantânea de QR Code após confirmação de pagamento
    \item Atualização automática do código a cada 30 segundos para evitar capturas de tela
    \item Vinculação criptografada entre ingresso e usuário
    \item Formato PDF para download e impressão quando necessário
    \item Taxa de fraude inferior a 0,1\% do volume total de ingressos
\end{itemize}

\textbf{5. Aplicação de Controle de Acesso (Check-in)}

Interface otimizada para operadores de portaria, com capacidade de validar ingressos através de leitura de QR Code, funcionar em modo offline com sincronização posterior, e fornecer feedback visual claro sobre status de validação. Resultados técnicos esperados:

\begin{itemize}
    \item Tempo de validação por ingresso inferior a 2 segundos
    \item Funcionamento offline com cache de até 50.000 ingressos
    \item Sincronização automática e inteligente ao restaurar conectividade
    \item Taxa de sucesso na validação superior a 99,5\%
    \item Interface responsiva otimizada para tablets e smartphones
\end{itemize}

\subsection{Arquitetura e Infraestrutura}

A plataforma será construída sobre uma arquitetura de microsserviços moderna e escalável, implantada em infraestrutura de nuvem com as seguintes características:

\textbf{Arquitetura de Microsserviços:}
\begin{itemize}
    \item 8 microsserviços independentes e fracamente acoplados
    \item Comunicação via APIs REST com documentação OpenAPI/Swagger
    \item Database per service pattern para independência de dados
    \item Event-driven architecture para processamentos assíncronos
    \item Containerização com Docker para portabilidade
\end{itemize}

\textbf{Infraestrutura Cloud:}
\begin{itemize}
    \item Ambientes segregados (desenvolvimento, homologação, produção)
    \item Auto-scaling horizontal baseado em métricas de carga
    \item Load balancing com distribuição inteligente de tráfego
    \item CDN global para distribuição de conteúdo estático
    \item Backup automatizado com RPO de 4 horas e RTO de 4 horas
\end{itemize}

\textbf{Segurança e Conformidade:}
\begin{itemize}
    \item Certificação PCI DSS Level 1 para processamento de pagamentos
    \item Conformidade total com LGPD para proteção de dados pessoais \cite{LGPD:2018}
    \item Criptografia TLS 1.3 para dados em trânsito
    \item Criptografia AES-256 para dados sensíveis em repouso
    \item Proteção contra todas as vulnerabilidades do OWASP Top 10 \cite{OWASP:2021}
    \item Web Application Firewall (WAF) configurado e operacional
\end{itemize}

\subsection{Qualidade de Software}

O projeto entregará um sistema com altos padrões de qualidade, evidenciados por:

\textbf{Cobertura de Testes:}
\begin{itemize}
    \item Cobertura de testes unitários superior a 80\%
    \item Suite completa de testes de integração cobrindo fluxos críticos
    \item Testes end-to-end automatizados para jornadas principais de usuário
    \item Validação de performance com simulação de 10.000 usuários simultâneos
    \item Auditoria de segurança externa sem vulnerabilidades críticas
\end{itemize}

\textbf{Métricas de Qualidade de Código:}
\begin{itemize}
    \item Índice de manutenibilidade classificação A no SonarQube
    \item Densidade de defeitos inferior a 0,5 por 1000 linhas de código
    \item Dívida técnica inferior a 5\% (classificação A)
    \item Zero code smells críticos ou bloqueadores
    \item Documentação de APIs 100\% completa e atualizada
\end{itemize}

\textbf{Performance e Disponibilidade:}
\begin{itemize}
    \item Tempo de resposta médio de APIs inferior a 200ms (P95 < 500ms)
    \item Tempo de carregamento de páginas inferior a 2 segundos
    \item Disponibilidade superior a 99,9\% (SLA de 8,76 horas downtime/ano)
    \item Taxa de erro de requisições inferior a 0,1\%
    \item Capacidade demonstrada de escalar para 10x carga normal
\end{itemize}

\section{Melhorias para Stakeholders}

\subsection{Benefícios para Organizadores de Eventos}

Os organizadores de eventos obterão uma plataforma confiável e profissional que otimiza significativamente suas operações de comercialização de ingressos:

\textbf{Eficiência Operacional:}
\begin{itemize}
    \item Redução de 70\% no tempo gasto com gestão manual de ingressos
    \item Eliminação de processos manuais de controle de estoque e vendas
    \item Dashboard centralizado com visibilidade em tempo real de todas as métricas
    \item Relatórios automatizados de vendas, público e receita
    \item Redução estimada de 50\% em custos operacionais versus bilheteria física
\end{itemize}

\textbf{Segurança e Controle:}
\begin{itemize}
    \item Redução de 90\% em casos de falsificação de ingressos
    \item Controle efetivo sobre cambismo através de vinculação usuário-ingresso
    \item Auditoria completa de todas as transações e acessos
    \item Prevenção de overselling através de controle de estoque em tempo real
    \item Mitigação de fraudes financeiras através de gateway certificado
\end{itemize}

\textbf{Capacidade e Alcance:}
\begin{itemize}
    \item Capacidade de vender para público global 24/7 sem limitações geográficas
    \item Escalabilidade para suportar eventos de qualquer porte (10 a 100.000 participantes)
    \item Integração com marketing digital através de links compartilháveis
    \item Aumento estimado de 30\% no alcance de público potencial
    \item Redução de 40\% no tempo entre lançamento e esgotamento de ingressos
\end{itemize}

\subsection{Benefícios para Compradores}

Os consumidores finais experimentarão uma jornada de compra moderna, segura e conveniente:

\textbf{Experiência do Usuário:}
\begin{itemize}
    \item Processo de compra simplificado com máximo 3 etapas
    \item Interface responsiva funcionando perfeitamente em qualquer dispositivo
    \item Busca e descoberta facilitada de eventos de interesse
    \item Confirmação instantânea e envio automático de ingressos por e-mail
    \item Acesso permanente aos ingressos através do portal do cliente
\end{itemize}

\textbf{Confiança e Segurança:}
\begin{itemize}
    \item Garantia de autenticidade dos ingressos através de QR Codes únicos
    \item Proteção de dados pessoais e financeiros conforme LGPD
    \item Processamento seguro de pagamentos certificado PCI DSS
    \item Transparência total sobre taxas, políticas e informações do evento
    \item Redução de 95\% no risco de aquisição de ingressos falsificados
\end{itemize}

\textbf{Conveniência:}
\begin{itemize}
    \item Compra a qualquer hora, de qualquer lugar, em qualquer dispositivo
    \item Eliminação de deslocamento a pontos de venda físicos
    \item Múltiplos métodos de pagamento (PIX instantâneo, cartão de crédito)
    \item Histórico completo de compras e ingressos em um único lugar
    \item Reenvio instantâneo de ingressos em caso de perda ou esquecimento
\end{itemize}

\subsection{Benefícios para Operadores de Portaria}

A equipe de controle de acesso contará com ferramentas eficientes que agilizam e profissionalizam o processo de entrada:

\textbf{Agilidade e Eficiência:}
\begin{itemize}
    \item Redução de 80\% no tempo médio de validação por participante
    \item Eliminação de filas extensas na entrada (processamento < 2 segundos)
    \item Interface intuitiva requerendo treinamento mínimo (< 15 minutos)
    \item Funcionamento offline garantindo operação mesmo com conectividade falha
    \item Redução estimada de 60\% na necessidade de equipe de portaria
\end{itemize}

\textbf{Segurança e Confiabilidade:}
\begin{itemize}
    \item Detecção instantânea de tentativas de ingresso duplicado ou fraudulento
    \item Registro completo de horários de entrada para auditoria
    \item Alertas visuais claros sobre status de validação (aprovado/rejeitado)
    \item Impossibilidade de manipulação manual de registros de entrada
    \item Sincronização automática de dados entre múltiplos pontos de acesso
\end{itemize}

\section{Impactos Projetados}

\subsection{Impacto no Setor de Eventos}

A plataforma desenvolvida tem potencial para elevar significativamente os padrões de qualidade e profissionalismo no setor de eventos brasileiro:

\textbf{Democratização do Acesso:}
\begin{itemize}
    \item Organizadores de eventos de pequeno e médio porte terão acesso a tecnologia antes disponível apenas para grandes players
    \item Redução de barreiras de entrada para novos organizadores através de custos operacionais menores
    \item Expansão estimada de 40\% no número de eventos que adotam comercialização digital profissional
\end{itemize}

\textbf{Profissionalização do Mercado:}
\begin{itemize}
    \item Estabelecimento de novos padrões de segurança e conformidade regulatória
    \item Redução de práticas predatórias como cambismo e falsificação
    \item Aumento da confiança do consumidor no mercado de eventos online
    \item Benchmark para outras soluções do setor em termos de usabilidade e segurança
\end{itemize}

\textbf{Dados e Inteligência:}
\begin{itemize}
    \item Geração de dados estruturados sobre perfil de público e comportamento de compra
    \item Possibilidade de análises preditivas para otimização de preços e estratégias
    \item Insights para organizadores melhorarem planejamento e experiência de eventos
\end{itemize}

\subsection{Impacto Acadêmico e Científico}

Do ponto de vista acadêmico, o projeto contribui significativamente para o conhecimento e formação dos envolvidos:

\textbf{Aplicação Prática de Conhecimentos:}
\begin{itemize}
    \item Consolidação de conceitos de arquitetura de software, microsserviços e cloud computing
    \item Aplicação real de metodologias ágeis e gestão de projetos
    \item Experiência prática com padrões de segurança e conformidade regulatória
    \item Desenvolvimento de competências em engenharia de requisitos e design centrado no usuário
\end{itemize}

\textbf{Contribuição para a Literatura:}
\begin{itemize}
    \item Documentação completa do ciclo de desenvolvimento de sistema web complexo
    \item Estudo de caso aplicável em disciplinas de engenharia de software
    \item Referência para trabalhos futuros sobre sistemas de e-commerce e eventos
    \item Validação empírica de práticas de desenvolvimento seguro e escalável
\end{itemize}

\textbf{Formação Profissional:}
\begin{itemize}
    \item Preparação da equipe para atuação no mercado de tecnologia
    \item Desenvolvimento de portfolio técnico robusto e demonstrável
    \item Experiência em todas as fases do ciclo de vida de desenvolvimento de software
    \item Habilidades de trabalho em equipe, comunicação e gestão de stakeholders
\end{itemize}

\subsection{Impacto Social}

Embora seja primariamente um projeto tecnológico, a plataforma traz benefícios sociais relevantes:

\textbf{Acessibilidade:}
\begin{itemize}
    \item Interface desenvolvida seguindo diretrizes WCAG 2.1 nível AA \cite{WCAG:2018}
    \item Possibilidade de participação em eventos para pessoas com limitações de mobilidade (compra online, sem necessidade de deslocamento a bilheteria)
    \item Democratização do acesso a informações sobre eventos culturais
\end{itemize}

\textbf{Transparência e Proteção ao Consumidor:}
\begin{itemize}
    \item Conformidade total com Código de Defesa do Consumidor
    \item Clareza de informações sobre preços, taxas e políticas de cancelamento
    \item Proteção contra fraudes e práticas abusivas de cambismo
    \item Garantia de privacidade e proteção de dados pessoais conforme LGPD
\end{itemize}

\textbf{Inclusão Digital:}
\begin{itemize}
    \item Incentivo à adoção de tecnologias digitais por organizadores tradicionais
    \item Educação de usuários sobre práticas seguras de compra online
    \item Contribuição para redução da exclusão digital no acesso a entretenimento e cultura
\end{itemize}

\section{Indicadores de Sucesso}

O sucesso do projeto será mensurado através dos seguintes indicadores-chave:

\subsection{Indicadores Técnicos}

\begin{table}[H]
\centering
\caption{Indicadores técnicos de sucesso do projeto.}
\label{tab:kpi-tecnicos}
\small
\begin{tabular}{|p{5cm}|p{2.5cm}|p{3cm}|p{2cm}|}
\hline
\textbf{Indicador} & \textbf{Meta} & \textbf{Método Medição} & \textbf{Período} \\ \hline
Disponibilidade do sistema & > 99,9\% & APM/Monitoramento & Contínuo \\ \hline
Tempo resposta P95 APIs & < 500ms & APM & Contínuo \\ \hline
Tempo carregamento páginas & < 2s & Lighthouse & Semanal \\ \hline
Cobertura testes unitários & > 80\% & SonarQube & Sprint \\ \hline
Vulnerabilidades críticas & 0 & SAST/DAST & Sprint \\ \hline
Taxa erro transações & < 0,1\% & Logs/Analytics & Diário \\ \hline
Capacidade concorrência & 10.000 users & Testes carga & Mensal \\ \hline
\end{tabular}
\end{table}

\subsection{Indicadores de Experiência do Usuário}

\begin{table}[H]
\centering
\caption{Indicadores de experiência do usuário.}
\label{tab:kpi-ux}
\small
\begin{tabular}{|p{5cm}|p{2.5cm}|p{3cm}|p{2cm}|}
\hline
\textbf{Indicador} & \textbf{Meta} & \textbf{Método Medição} & \textbf{Período} \\ \hline
Taxa conclusão compra & > 95\% & Analytics & Diário \\ \hline
Tempo médio compra & < 3 min & Analytics & Diário \\ \hline
Taxa abandono carrinho & < 15\% & Analytics & Diário \\ \hline
System Usability Scale (SUS) & > 80 & Pesquisa & Trimestral \\ \hline
Net Promoter Score (NPS) & > 50 & Pesquisa & Trimestral \\ \hline
Satisfação pós-compra & > 4,5/5 & Pesquisa & Transação \\ \hline
Conformidade WCAG 2.1 AA & 100\% & Auditoria & Mensal \\ \hline
\end{tabular}
\end{table}

\subsection{Indicadores de Negócio}

\begin{table}[H]
\centering
\caption{Indicadores de negócio e adoção.}
\label{tab:kpi-negocio}
\small
\begin{tabular}{|p{5cm}|p{2.5cm}|p{3cm}|p{2cm}|}
\hline
\textbf{Indicador} & \textbf{Meta} & \textbf{Método Medição} & \textbf{Período} \\ \hline
Organizadores cadastrados & 50 & Banco dados & 6 meses \\ \hline
Eventos publicados & 100 & Banco dados & 12 meses \\ \hline
Ingressos vendidos & 100.000 & Banco dados & 12 meses \\ \hline
Taxa sucesso pagamentos & > 99\% & Gateway/Logs & Diário \\ \hline
Taxa sucesso validações & > 99,5\% & Logs check-in & Evento \\ \hline
Crescimento trimestral & 25\% & Banco dados & Trimestre \\ \hline
Taxa fraude detectada & < 0,1\% & ML/Analytics & Mensal \\ \hline
\end{tabular}
\end{table}

\section{Sustentabilidade e Evolução Futura}

Embora o projeto tenha prazo definido de 30 semanas, espera-se que a plataforma desenvolvida tenha sustentabilidade de longo prazo:

\textbf{Documentação Completa:}
\begin{itemize}
    \item Documentação técnica detalhada de arquitetura, APIs e deploy
    \item Manuais de usuário para cada perfil (comprador, organizador, operador)
    \item Runbooks para operação, troubleshooting e resolução de incidentes
    \item Base de conhecimento para suporte e manutenção
\end{itemize}

\textbf{Código Limpo e Manutenível:}
\begin{itemize}
    \item Arquitetura modular facilitando evolução independente de componentes
    \item Padrões de código consistentes e bem documentados
    \item Baixa dívida técnica e alta cobertura de testes automatizados
    \item Facilidade para onboarding de novos desenvolvedores
\end{itemize}

\textbf{Roadmap de Evolução:}

Funcionalidades candidatas para fases futuras pós-projeto inicial:
\begin{itemize}
    \item \textbf{Fase 2 (3-6 meses):} Marketplace secundário para revenda segura, escolha de assentos marcados, programa de fidelidade
    \item \textbf{Fase 3 (6-12 meses):} Aplicativos móveis nativos iOS/Android, analytics avançado com machine learning, integração com redes sociais
    \item \textbf{Fase 4 (12+ meses):} Expansão internacional, suporte multi-idioma e multi-moeda, white-label para organizadores grandes
\end{itemize}

\section{Considerações Finais}

Os resultados esperados apresentados neste capítulo demonstram que o projeto de desenvolvimento da plataforma de ingressos online não se limita à entrega de um produto tecnológico, mas representa uma solução abrangente que endereça desafios reais do setor de eventos, beneficia múltiplos stakeholders e tem potencial de impacto significativo no mercado.

O alinhamento rigoroso entre objetivos definidos, planejamento detalhado, recursos adequados e métricas de sucesso claras estabelece bases sólidas para o atingimento dos resultados projetados. A ênfase em qualidade, segurança, usabilidade e escalabilidade posiciona a plataforma como uma referência técnica e funcional no domínio de sistemas de comercialização de ingressos online.

Mais do que um trabalho acadêmico, espera-se que este projeto contribua efetivamente para a evolução do setor de eventos no Brasil, promovendo profissionalização, segurança e melhoria da experiência de todos os envolvidos no ecossistema de eventos culturais, esportivos e corporativos.

