%% USPSC-TCC-pre-textual-OUTROS.tex
%% Camandos para definição do tipo de documento (tese ou dissertação), área de concentração, opção, preâmbulo, titulação 
%% referentes aos Programas de Pós-Graduação
\instituicao{Engenharia da Computa\c{c}\~ao, Centro Universit\'ario de Vi\c{c}osa}
\unidade{AN\'ALISE E DESENVOLVIMENTO DE SISTEMAS}
\unidademin{An\'alise e Desenvolvimento de Sistemas}
\universidademin{Centro Universit\'ario de Vi\c{c}osa}

% A EESC não inclui a nota "Versão original", portanto o comando abaixo não tem a mensagem entre {}

\notafolharosto{ }
%Para a versão corrigida tire a % do comando/declaração abaixo e inclua uma % antes do comando acima
%\notafolharosto{VERS\~AO CORRIGIDA}
% ---
% dados complementares para CAPA e FOLHA DE ROSTO
% ---
\universidade{CENTRO UNIVERSIT\'ARIO DE VI\c{C}OSA}

\justifying
\titulo{Sistema Web Escalável para Venda de Ingressos Online}

\justifying
\titleabstract{Practical Work - ADS 504 - Software Architecture and Integrated Project}

\justifying
\tituloresumo{Trabalho Prático - ADS 504 – Arquitetura de Software e
Projeto Integrador}

\autor{EDSON RAMOS DA SILVA JUNIOR\\BERNARDO CORDEIRO MOTTA}
\autorficha{Junior, edson\\motta, bernardo}
\autorabr{JUNIOR, Edson\\MOTTA, Bernardo}

\cutter{S856m}
% Para gerar a ficha catalográfica sem o Código Cutter, basta 
% incluir uma % na linha acima e tirar a % da linha abaixo
%\cutter{ }

\local{Vi\c{c}osa}
\data{2025}
% Quando for Orientador, basta incluir uma % antes do comando abaixo
\renewcommand{\orientadorname}{Orientadora: Carlos Henrique Tavares Brumatti}
% Quando for Coorientadora, basta tirar a % do comando abaixo
%\newcommand{\coorientadorname}{Coorientador:}
\orientador{Prof. Me Dr Dr Carlos Henrique Tavares Brumatti}
%\orientadorcorpoficha{orientadora Elisa Gon\c{c}alves Rodrigues}
%\orientadorficha{Rodrigues, Elisa Gon\c{c}alves, orient}
%Se houver co-orientador, inclua % antes das duas linhas (antes dos comandos \orientadorcorpoficha e \orientadorficha) 
%          e tire a % antes dos 3 comandos abaixo
\coorientador{título e coorientador aqui}
\orientadorcorpoficha{orientadora \red{nome do orientador} ;  co-orientador \red{nome do coorientador}}
\orientadorficha{Lana, Cristiane Aparecida, orient. II. Milton Miranda Neto, co-orient}

%\notaautorizacao{AUTORIZO A REPRODU\c{C}\~AO E DIVULGA\c{C}\~AO TOTAL OU PARCIAL DESTE TRABALHO, POR QUALQUER MEIO CONVENCIONAL OU ELETR\^ONICO PARA FINS DE ESTUDO E PESQUISA, DESDE QUE CITADA A FONTE.}
\notabib{~  ~}

\newcommand{\programa}[1]{     	
% Outros
	\tipotrabalho{Plano de Projeto de Conclus\~ao de Curso}
	\tipotrabalhoabs{Plan Project of Conclusion Course)}
	%\area{Nome da área}
	%\opcao{Nome da Opção}
	% O preambulo deve conter o tipo do trabalho, o objetivo, 
	% o nome da instituição, a área de concentração e opção quando houver
	\preambulo{Plano de Projeto apresentado ao Curso de An\'alise e Desenvolvimento de Sistemas do Centro Universit\'ario de Vi\c{c}osa, como parte dos requisitos para conclus\~ao da disciplina ADS504 Arquitetura de Software e Projeto Integrador.}
	\notaficha{Projeto de (Tecnólogo em An\'alise e Desenvolvimento de Sistemas}	
	}

