\begin{resumo}

JUNIOR, Edson Ramos da Silva; MOTTA, Bernardo Cordeiro. \textbf{Plataforma Web Escalável e Segura para Venda de Ingressos Online: Arquitetura, Requisitos e Implementação}. 2025. Plano de Projeto — Curso de Análise e Desenvolvimento de Sistemas, Centro Universitário de Viçosa, Viçosa, 2025.

A indústria global de eventos enfrenta desafios significativos na comercialização digital de ingressos, incluindo problemas de escalabilidade durante picos de demanda, vulnerabilidades de segurança e experiências de usuário inadequadas. Este trabalho apresenta o projeto arquitetural de uma plataforma web robusta para venda de ingressos online, fundamentada em princípios de engenharia de software e centrada nas necessidades dos stakeholders. A metodologia adotada incluiu análise detalhada dos stakeholders através de personas e mapa de empatia, seguida pela especificação completa de requisitos funcionais e não funcionais. A solução proposta contempla arquitetura de microsserviços com infraestrutura em nuvem escalável, sistema de autenticação seguro, gestão completa de eventos, fluxo de compra otimizado com integração a gateways de pagamento, geração de ingressos digitais com QR Codes dinâmicos e sistema de controle de acesso com funcionamento offline. O projeto estabelece conformidade com padrões de segurança PCI DSS, proteção contra vulnerabilidades OWASP Top 10, aderência à LGPD e acessibilidade segundo WCAG 2.1. A análise de riscos e restrições garante viabilidade técnica e operacional da solução, visando elevar os padrões de qualidade e confiabilidade no setor de eventos digitais.

\textbf{Palavras-chave:} sistemas de ingressos; arquitetura de software; microsserviços; segurança da informação; experiência do usuário; escalabilidade.

\end{resumo}
